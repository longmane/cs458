\documentclass[journal]{vgtc}                % final (journal style)
%\documentclass[review,journal]{vgtc}         % review (journal style)
%\documentclass[widereview]{vgtc}             % wide-spaced review
%\documentclass[preprint,journal]{vgtc}       % preprint (journal style)

%% Uncomment one of the lines above depending on where your paper is
%% in the conference process. ``review'' and ``widereview'' are for review
%% submission, ``preprint'' is for pre-publication, and the final version
%% doesn't use a specific qualifier.

%% Please use one of the ``review'' options in combination with the
%% assigned online id (see below) ONLY if your paper uses a double blind
%% review process. Some conferences, like IEEE Vis and InfoVis, have NOT
%% in the past.

%% Please note that the use of figures other than the optional teaser is not permitted on the first page
%% of the journal version.  Figures should begin on the second page and be
%% in CMYK or Grey scale format, otherwise, colour shifting may occur
%% during the printing process.  Papers submitted with figures other than the optional teaser on the
%% first page will be refused. Also, the teaser figure should only have the
%% width of the abstract as the template enforces it.

%% These few lines make a distinction between latex and pdflatex calls and they
%% bring in essential packages for graphics and font handling.
%% Note that due to the \DeclareGraphicsExtensions{} call it is no longer necessary
%% to provide the the path and extension of a graphics file:
%% \includegraphics{diamondrule} is completely sufficient.
%%
\ifpdf%                                % if we use pdflatex
  \pdfoutput=1\relax                   % create PDFs from pdfLaTeX
  \pdfcompresslevel=9                  % PDF Compression
  \pdfoptionpdfminorversion=7          % create PDF 1.7
  \ExecuteOptions{pdftex}
  \usepackage{graphicx}                % allow us to embed graphics files
  \DeclareGraphicsExtensions{.pdf,.png,.jpg,.jpeg} % for pdflatex we expect .pdf, .png, or .jpg files
\else%                                 % else we use pure latex
  \ExecuteOptions{dvips}
  \usepackage{graphicx}                % allow us to embed graphics files
  \DeclareGraphicsExtensions{.eps}     % for pure latex we expect eps files
\fi%

%% it is recomended to use ``\autoref{sec:bla}'' instead of ``Fig.~\ref{sec:bla}''
\graphicspath{{figures/}{pictures/}{images/}{./}} % where to search for the images

\usepackage{microtype}                 % use micro-typography (slightly more compact, better to read)
\PassOptionsToPackage{warn}{textcomp}  % to address font issues with \textrightarrow
\usepackage{textcomp}                  % use better special symbols
\usepackage{mathptmx}                  % use matching math font
\usepackage{times}                     % we use Times as the main font
\renewcommand*\ttdefault{txtt}         % a nicer typewriter font
\usepackage{cite}                      % needed to automatically sort the references
%\usepackage{tabu}                      % only used for the table example
%\usepackage{booktabs}                  % only used for the table example
%% We encourage the use of mathptmx for consistent usage of times font
%% throughout the proceedings. However, if you encounter conflicts
%% with other math-related packages, you may want to disable it.

%% In preprint mode you may define your own headline.
%\preprinttext{To appear in IEEE Transactions on Visualization and Computer Graphics.}

%% If you are submitting a paper to a conference for review with a double
%% blind reviewing process, please replace the value ``0'' below with your
%% OnlineID. Otherwise, you may safely leave it at ``0''.
\onlineid{0}

%% declare the category of your paper, only shown in review mode
\vgtccategory{Research}
%% please declare the paper type of your paper to help reviewers, only shown in review mode
%% choices:
%% * algorithm/technique
%% * application/design study
%% * evaluation
%% * system
%% * theory/model
\vgtcpapertype{evaluation}

%% Paper title.
\title{Improvements to a Linguistic Approach to Color Assignment}

%% This is how authors are specified in the journal style

%% indicate IEEE Member or Student Member in form indicated below
\author{Mikky Cecil, Emily Longman, and Jacob Lewis}

%other entries to be set up for journal
\shortauthortitle{Biv \MakeLowercase{\textit{et al.}}: Improvements to a Linguistic Approach to Color Assignment}
%\shortauthortitle{Firstauthor \MakeLowercase{\textit{et al.}}: Paper Title}

%% Abstract section.
\abstract{  
} % end of abstract

%% Keywords that describe your work. Will show as 'Index Terms' in journal
%% please capitalize first letter and insert punctuation after last keyword
\keywords{information visualization, linguistics, color theory, linguistic relativity}

%% ACM Computing Classification System (CCS). 
%% See <http://www.acm.org/class/1998/> for details.
%% The ``\CCScat'' command takes four arguments.

\CCScatlist{ % not used in journal version
 \CCScat{K.6.1}{Management of Computing and Information Systems}%
{Project and People Management}{Life Cycle};
 \CCScat{K.7.m}{The Computing Profession}{Miscellaneous}{Ethics}
}

%% Uncomment below to include a teaser figure.

%% Uncomment below to disable the manuscript note
%\renewcommand{\manuscriptnotetxt}{}

%% Copyright space is enabled by default as required by guidelines.
%% It is disabled by the 'review' option or via the following command:
% \nocopyrightspace

\vgtcinsertpkg

%%%%%%%%%%%%%%%%%%%%%%%%%%%%%%%%%%%%%%%%%%%%%%%%%%%%%%%%%%%%%%%%
%%%%%%%%%%%%%%%%%%%%%% START OF THE PAPER %%%%%%%%%%%%%%%%%%%%%%
%%%%%%%%%%%%%%%%%%%%%%%%%%%%%%%%%%%%%%%%%%%%%%%%%%%%%%%%%%%%%%%%%

\begin{document}

%% The ``\maketitle'' command must be the first command after the
%% ``\begin{document}'' command. It prepares and prints the title block.

%% the only exception to this rule is the \firstsection command
%%\firstsection{Introduction}

\maketitle

\section{Introduction} %for journal use above \firstsection{..} instead
Quickly generating color palettes for categorical information is a challenge in InfoViz, particularly if the values already have strong color associations (for example, if one had a list of fruits, readers may be thrown off if a pie chart was cut into an orange section labeled ``apples,'' and a red section labeled ``oranges''). 
The paper ``A Linguistic Approach to Categorical Color Assignment
for Data Visualization'' by Vidya Setlur and Maureen C. Stoneaimed \cite{basis}
explores an algorithm which used to use natural language processing in order to generate color palettes, building off of (and comparing to) prior research. 
Like some prior research, they wanted to use Google image searches and statistics to discern whether or not a word had a strong color association with it, and if so, what color best represented that word. What was different about this research was that it also applied contextual information to the colors---for example, the word ``apple'' could either mean a red fruit or a silver tech brand, depending on the context.

\section{Related Work}
The subject color differentiation has always been a big one.
Many different researchers in a variety of fields have attempted to work on this.
This specific topic in InfoViz doesn't have much previous research, but the field of linguistic relativity has investigated it.
Many looked at it from a linguistic, cultural, or anthropological perspective, including a paper by Terry Regier and Paul Kay which asserted that ``1, language influences color perception primarily in half
the visual field, and 2, color naming across languages is
shaped by both universal and language-specific forces'' \cite{Whorf}

A different paper from a group of Psychology researchers ran three different experiments on subjects' ability to differentiate colors, if practiced helped this, and how hue and lightness affected this.
The results of this lend themselves to the theory that language does have an effect on color perception. \cite{ozgen}

A number of different strategies have been implemented in more computerized studies, such as using color palettes designed by experts, sourcing images from Google and using n-grams, and even scraping Flickr annotations.
The researchers from the paper off which we're building used all of these previous attempts to test the validity of their own methods and algorithms.


\subsection{Challenges in Visualization}
One of the biggest problems that this study faced was the light to dark variance in some of the final palettes.
Legibility and harmonic differences made them less than consistent.
The goal of our project was to adjust the way these were generated so that these negative side effects could be minimized.
We did not have access to the actual algorithm used by the researchers, so this meant we had to get creative with our methods.


\section{Research Design}
The basic steps involved in our solution to reducing errors in the output colors are:
\begin{itemize}
  \item Augment the algorithm that the authors of the paper created
  \item Perform multiple searches for same color with adjectives such as \'light\', \'pale\', \'deep\', or \'dark\'
  \item Throw out outliers and find average of these colors
  \item Obtain results that should reduce the errors that the author saw with dark colors
\end{itemize}

\section{Methods}

\subsection{Method Overview}
There was a general methodology followed to produce our improved palette outputs.
First colors were searched on google using four adjectives: light, pale, deep, dark.
Next we collected average color of images using a color averaging tool found on github. \cite{avg}
Then the outliers were thrown out and an average was found for all four adjectives.
A color palette from each of these adjectives for each color was then created.
Next we found standard color name conventions from schemes including X11 and HTML4.
Then the difference between our color and the standard was calculated.
Lastly we compared our difference to that of the authors of the original paper’s difference.

\subsection{Visualization Tasks}


\subsection{Data Sources}


\subsection{Design Comparison}


\section{Implementation}
Code we used and specifics of getting it to work goes here.


\section{Results and Performance}


\section{Conclusions and Future Improvements}


%\bibliographystyle{abbrv}
\bibliographystyle{abbrv-doi}
%\bibliographystyle{abbrv-doi-narrow}
%\bibliographystyle{abbrv-doi-hyperref}
%\bibliographystyle{abbrv-doi-hyperref-narrow}

\bibliography{color_viz}
\end{document}